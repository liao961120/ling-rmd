%加這個就可以設定字體
\usepackage{fontspec}
%使用xeCJK,其他的還有CJK或是xCJK
\usepackage{xeCJK}
\usepackage{bm}

% Pandoc 只能設置 10, 11, 12 pt 超出此範圍需用此指令
%\usepackage[fontsize=13pt]{scrextend}

%設定英文字型,不設的話就會使用預設的字型
\setmainfont{Liberation Serif}

%設定中英文的字型
%字型的設定可以使用系統內的字型,而不用像以前一樣另外安裝
% AR PL KaitiM Big5 標楷體
% .PingFang TC
% Noto Sans CJK TC
% HanWangHeiLight
% HanWangHeiHeavy
\setCJKmainfont[
	BoldFont={HanWangHeiHeavy}  % 粗體字使用華康粗明體
    ]{HanWangHeiLight}    % 一般字型則使用華康細明體

%中文自動換行
\XeTeXlinebreaklocale "zh"

%文字的彈性間距
\XeTeXlinebreakskip = 0pt plus 1pt

%設定段落之間的距離
\setlength{\parskip}{0.15cm}

%設定行距
\linespread{1.2}\selectfont
